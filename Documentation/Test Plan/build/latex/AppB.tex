\chapter{Issue Tracking Manager}

ZenHub is an agile project management application within GitHub that allows organization of issues into task boards and reports.


\subsection{Installation}
\label{\detokenize{test_plan/issue_tracking_manager:installation}}
Download the browser extension for Firefox or Chrome and accept permissions and sign in with GitHub. ZenHub will be added right into the GitHub interface.


\subsection{Structure}
\label{\detokenize{test_plan/issue_tracking_manager:structure}}
The ZenHub Workspace is split up between 7 pipelines;
\begin{itemize}
\item {} \begin{description}
\item[{New Issues}] \leavevmode
Issues that have yet to be assigned or evaluated.

\end{description}

\item {} \begin{description}
\item[{Icebox}] \leavevmode
Features that are desired but not required.

\end{description}

\item {} \begin{description}
\item[{Backlog}] \leavevmode
Issues that were not completed from the past Sprint. The highest priority.

\end{description}

\item {} \begin{description}
\item[{In Progress}] \leavevmode
Issues that currently being worked on by the Development team.

\end{description}

\item {} \begin{description}
\item[{Review/QA}] \leavevmode
Issues that are ready to be code reviewed and QA tested.

\end{description}

\item {} \begin{description}
\item[{Done}] \leavevmode
Issues that have passed review and QA tests and can be queued for next release.

\end{description}

\item {} \begin{description}
\item[{Closed}] \leavevmode
Issues that will no longer be worked on (Done, won’t fix, done with development, already implemented, duplicate, etc.).

\end{description}

\end{itemize}


\subsection{Issue Types}
\label{\detokenize{test_plan/issue_tracking_manager:issue-types}}\begin{itemize}
\item {} 
User Story

\item {} 
Feature Request

\item {} 
UX Issue

\item {} 
Performance

\item {} 
Bug

\end{itemize}


\subsection{Issue Complexities}
\label{\detokenize{test_plan/issue_tracking_manager:issue-complexities}}

\begin{savenotes}\sphinxattablestart
\centering
\begin{tabulary}{\linewidth}[t]{|T|T|}
\hline
\sphinxstyletheadfamily 
Complexity
&\sphinxstyletheadfamily 
Difficulty
\\
\hline
1
&
Trivial (Typo, small fix)
\\
\hline
2
&
Busy Work (Easy Function)
\\
\hline
3
&
Rare (New Function)
\\
\hline
5
&
Medium Rare (Set of interacting functions)
\\
\hline
8
&
Medium Well (New feature)
\\
\hline
13
&
Well Done (Major fix, multiple features)
\\
\hline
21
&
Perplexing (Researching new techniques)
\\
\hline
40
&
Beastly (Unsure if possible)
\\
\hline
\end{tabulary}
\par
\sphinxattableend\end{savenotes}



\renewcommand{\indexname}{Index}
\printindex