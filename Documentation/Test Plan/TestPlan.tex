%% Generated by Sphinx.
\def\sphinxdocclass{report}
\documentclass[letterpaper,10pt,english,openany,oneside]{sphinxmanual}
\ifdefined\pdfpxdimen
   \let\sphinxpxdimen\pdfpxdimen\else\newdimen\sphinxpxdimen
\fi \sphinxpxdimen=.75bp\relax

\PassOptionsToPackage{warn}{textcomp}
\usepackage[utf8]{inputenc}
\ifdefined\DeclareUnicodeCharacter
% support both utf8 and utf8x syntaxes
  \ifdefined\DeclareUnicodeCharacterAsOptional
    \def\sphinxDUC#1{\DeclareUnicodeCharacter{"#1}}
  \else
    \let\sphinxDUC\DeclareUnicodeCharacter
  \fi
  \sphinxDUC{00A0}{\nobreakspace}
  \sphinxDUC{2500}{\sphinxunichar{2500}}
  \sphinxDUC{2502}{\sphinxunichar{2502}}
  \sphinxDUC{2514}{\sphinxunichar{2514}}
  \sphinxDUC{251C}{\sphinxunichar{251C}}
  \sphinxDUC{2572}{\textbackslash}
\fi
\usepackage{cmap}
\usepackage[T1]{fontenc}
\usepackage{amsmath,amssymb,amstext}
\usepackage{babel}



\usepackage{times}
\expandafter\ifx\csname T@LGR\endcsname\relax
\else
% LGR was declared as font encoding
  \substitutefont{LGR}{\rmdefault}{cmr}
  \substitutefont{LGR}{\sfdefault}{cmss}
  \substitutefont{LGR}{\ttdefault}{cmtt}
\fi
\expandafter\ifx\csname T@X2\endcsname\relax
  \expandafter\ifx\csname T@T2A\endcsname\relax
  \else
  % T2A was declared as font encoding
    \substitutefont{T2A}{\rmdefault}{cmr}
    \substitutefont{T2A}{\sfdefault}{cmss}
    \substitutefont{T2A}{\ttdefault}{cmtt}
  \fi
\else
% X2 was declared as font encoding
  \substitutefont{X2}{\rmdefault}{cmr}
  \substitutefont{X2}{\sfdefault}{cmss}
  \substitutefont{X2}{\ttdefault}{cmtt}
\fi


\usepackage[Bjornstrup]{fncychap}
\usepackage{sphinx}

\fvset{fontsize=\small}
\usepackage{geometry}

% Include hyperref last.
\usepackage{hyperref}
% Fix anchor placement for figures with captions.
\usepackage{hypcap}% it must be loaded after hyperref.
% Set up styles of URL: it should be placed after hyperref.
\urlstyle{same}

\usepackage{sphinxmessages}
\setcounter{tocdepth}{2}


    \usepackage{colortbl}
    \protected\def\sphinxstyletheadfamily{\cellcolor{gray}\sffamily}
    

\title{Test Plan Pangeo by VRtualize}
\date{Nov 13, 2019}
\release{2.0}
\author{Isaac Egermier\\Chezka Gaddi\\Garfield Tong\\Michael Theesen}
\newcommand{\sphinxlogo}{\vbox{}}
\renewcommand{\releasename}{Release}
\makeindex
\begin{document}

\pagestyle{empty}
\sphinxmaketitle
\pagestyle{plain}
\sphinxtableofcontents
\pagestyle{normal}
\phantomsection\label{\detokenize{test_plan::doc}}


\sphinxstylestrong{Revision History}


\begin{savenotes}\sphinxattablestart
\centering
\begin{tabular}[t]{|\X{10}{85}|\X{15}{85}|\X{15}{85}|\X{45}{85}|}
\hline
\sphinxstyletheadfamily 
Version
&\sphinxstyletheadfamily 
Date
&\sphinxstyletheadfamily 
Author
&\sphinxstyletheadfamily 
Description of Change
\\
\hline
1
&
October 17, 2019
&
Chezka Gaddi
&
First Draft
\\
\hline
2
&
October 25, 2019
&
Chezka Gaddi
&
Added QA testing, branch structure, issue life-cylce, and project timeline; changed issue management tool.
\\
\hline
\end{tabular}
\par
\sphinxattableend\end{savenotes}


\chapter{Introduction}
\label{\detokenize{test_plan/intro:introduction}}\label{\detokenize{test_plan/intro::doc}}

\section{Purpose}
\label{\detokenize{test_plan/intro:purpose}}
This test plan contains a description of the testing approach used to create a comprehensive plan for the testing of the Pangeo application which will be provided by the VRtualize team. This document includes:
\begin{itemize}
\item {} 
Strategy: Test rules and project assumptions including test objectives and certifications; end-to-end description of test set up.

\item {} 
Implementation: How testing is performed, defect management process (reporting, fix planning, and execution).

\item {} 
Management: Description of the logistics of testing.

\end{itemize}


\section{Overview}
\label{\detokenize{test_plan/intro:overview}}
The SDSMT VRtualize Team is working together with L3Harris on Project Pangeo. Pangeo is a research project aiming to render the real world in a dynamically loaded virtual reality environment. There are three main goals that contribute to this overall project: imagery retrieval, image caching, and virtual rendering.


\section{Audience}
\label{\detokenize{test_plan/intro:audience}}
The intended audience of the test plan is the Project Manager, Development team, and QA team. Some portions of this document may on occasion be shared with the client/user and other stakeholder whose input/approval into the testing process is needed.


\chapter{Strategy}
\label{\detokenize{test_plan/strategy:strategy}}\label{\detokenize{test_plan/strategy:id1}}\label{\detokenize{test_plan/strategy::doc}}

\section{Objectives}
\label{\detokenize{test_plan/strategy:objectives}}
Test objectives are meant to verify that the Pangeo application meets design specifications.

Testing will include the execution of automated tests, test scripts, and performance tests. Defects will be discussed with the team during the weekly stand-ups and prioritized as High, Medium, or Low severity. High and Medium severity defects must be retested as per the acceptance criteria. Low severity defects will be deferred to a subsequent sprint as discussed.
Criteria for the cessation of testing:
\begin{itemize}
\item {} 
Production ready  software (as per the Software Requirements Specification (SRS))

\item {} 
Automated tests and test scripts suitable for reuse as functional and user-acceptance testing.

\end{itemize}


\section{Assumptions}
\label{\detokenize{test_plan/strategy:assumptions}}
\sphinxstylestrong{General}
\begin{itemize}
\item {} 
Environment downtime will adversely impact test schedules.

\item {} 
Test environment will exactly duplicate the production environment.

\item {} 
Issue reporting includes complete reproduction details (as per Issue Reporting Template).

\item {} 
Issues are tracked using the ZenHub Issue Tracking System only.

\item {} 
Issues reported fixed after Certification N will include regression tests which will be added to the test plan for Certification N+1.

\end{itemize}

\sphinxstylestrong{Key Assumptions}
\begin{itemize}
\item {} 
Certifications are defined as follows:

\end{itemize}


\begin{savenotes}\sphinxattablestart
\centering
\begin{tabulary}{\linewidth}[t]{|T|T|}
\hline
\sphinxstyletheadfamily 
Certification
&\sphinxstyletheadfamily 
Description
\\
\hline
Certification 1a
&
Pulling data from a file and rendering an object that resembles the data.
\\
\hline
Certification 1b
&
Pulling data from a database instead of a file system.
\\
\hline
Certification 2
&
User interface of menu selections.
\\
\hline
Certification 3a
&
Pulling from a cache when necessary and pulling from the database ahead of the time.
\\
\hline
Certification 3b
&
User experience with movement.
\\
\hline
\end{tabulary}
\par
\sphinxattableend\end{savenotes}
\begin{itemize}
\item {} 
A release cannot go into production with any severity 1 (Critical), 2 (High) or 3 (Medium) defects.

\item {} 
Functional testing requires production-like data.

\item {} 
Alpha tests will be performed by identified alpha testers.

\end{itemize}


\section{Testing Scope}
\label{\detokenize{test_plan/strategy:testing-scope}}

\subsection{Unit}
\label{\detokenize{test_plan/strategy:unit}}
\sphinxstylestrong{Who:} Development Team

\sphinxstylestrong{When:} During product development

\sphinxstylestrong{Why:} Primarily for the purpose of identifying bugs at the unit level as early as possible

\sphinxstylestrong{Scope:} User stories, separate modular functions, scripts

\sphinxstylestrong{How:} All major software components will be developed using Unity’s built-in Test Runner which utilizes the NUnit framework or the Python module, unittest


\subsection{Back-end}
\label{\detokenize{test_plan/strategy:back-end}}
\sphinxstylestrong{Who:} Development Team

\sphinxstylestrong{When:} During database setup

\sphinxstylestrong{Why:} To avoid complications like deadlock, data corruption, and data loss

\sphinxstylestrong{Scope:} Database

\sphinxstylestrong{How:} Tests will be developed as SQL queries


\subsection{Code Reviews}
\label{\detokenize{test_plan/strategy:code-reviews}}
\sphinxstylestrong{Who:} Development Team

\sphinxstylestrong{When:} Within 4 days of an issue going into Review/QA pipeline

\sphinxstylestrong{Why:} To ensure that the code upholds coding standards

\sphinxstylestrong{Scope:} All units of code

\sphinxstylestrong{How:} Team members working on the same Certification will be required to participate and approve code reviews with the author of the code. Results in one of three results:
\begin{itemize}
\item {} 
Pass: An ok to push to merge with next branch

\item {} 
Revision/Pass: Can be pushed after some changes

\item {} 
Fail: Too many problems, will require another code review

\end{itemize}


\subsection{QA Testing}
\label{\detokenize{test_plan/strategy:qa-testing}}
\sphinxstylestrong{Who:} QA Team

\sphinxstylestrong{When:} After units of code passes code reviews

\sphinxstylestrong{Why:} To ensure units of code pass user acceptance criteria

\sphinxstylestrong{Scope:} All units of code

\sphinxstylestrong{How:} Using the testing software used to create the unit, back-end, and user acceptance tests


\subsection{Integration}
\label{\detokenize{test_plan/strategy:integration}}
\sphinxstylestrong{Who:} Development Team

\sphinxstylestrong{When:} Combining individual units of code

\sphinxstylestrong{Why:} To expose defects in the interfaces and interactions between integrated components or systems

\sphinxstylestrong{Scope:} Interaction between the database and the application, simulating the key interactions of a user using the application

\sphinxstylestrong{How:} Tests will be developed using the Integration Test Framework, which is part of the Unity Test Tools package


\subsection{System and Functional}
\label{\detokenize{test_plan/strategy:system-and-functional}}
\sphinxstylestrong{Who:} QA Team

\sphinxstylestrong{When:} Prior to Exploratory Testing

\sphinxstylestrong{Why:} Thorough testing of all application functions

\sphinxstylestrong{Scope:} All required features of the application as described in the SRS

\sphinxstylestrong{How:} Tests are performed using scripts, automated processes, and input decks.


\subsection{Soak and Performance}
\label{\detokenize{test_plan/strategy:soak-and-performance}}
\sphinxstylestrong{Who:} Development Team

\sphinxstylestrong{When:} Any new system update

\sphinxstylestrong{Why:} To ensure the application does not have any memory leaks and performs to the agreed-upon performance specification

\sphinxstylestrong{Scope:} Memory management, algorithms, and loading time

\sphinxstylestrong{How:} Unity Profiler and Unity Performance Testing Extension to internally monitor performance and optimizations of key systems.


\subsection{Stress}
\label{\detokenize{test_plan/strategy:stress}}
\sphinxstylestrong{Who:} Development Team

\sphinxstylestrong{When:} Before product release

\sphinxstylestrong{Why:} To determine the acceptable user limitations

\sphinxstylestrong{Scope:} Algorithms and loading time

\sphinxstylestrong{How:} Unity Profiler and Unity Performance Testing Extension to internally monitor performance and optimizations of key systems


\subsection{Exploratory and Alpha}
\label{\detokenize{test_plan/strategy:exploratory-and-alpha}}
\sphinxstylestrong{Who:} Alpha testers

\sphinxstylestrong{When:} After functional tests

\sphinxstylestrong{Why:} Primarily to familiarize the alpha testers with the features and behavior of the software to set expectations for new features and identify any hiccups

\sphinxstylestrong{Scope:} Production level product

\sphinxstylestrong{How:} Testers are encouraged to try the interface without scripts or documentation

\sphinxstylestrong{Deliverables:} UAT Test Cases written by Development Team and reviewed and signed off on by Development Team and Project Manager


\chapter{Issue Management}
\label{\detokenize{test_plan/issue_management:issue-management}}\label{\detokenize{test_plan/issue_management:id1}}\label{\detokenize{test_plan/issue_management::doc}}
\noindent\sphinxincludegraphics[width=600\sphinxpxdimen]{{issue_lifecycle}.png}


\section{Issue Life-Cycle}
\label{\detokenize{test_plan/issue_management:issue-life-cycle}}\begin{enumerate}
\sphinxsetlistlabels{\arabic}{enumi}{enumii}{}{.}%
\item {} 
Issue Creation: Project Manager assigns initial priority, effort, queue position, etc. and assign to a group based on Certification subsection.

\item {} 
Issue Assignment: During Sprint Planning, issues to be worked on get decided and are assigned to specific members.

\item {} 
Work on Issue: Develop Unit Tests at this time.

\item {} 
Issue goes through unit tests

\item {} 
Issue goes through code review: Code Review is scheduled no later than 4 days after the issue is moved into the QA/Review pipeline. Team members working on the same Certification subsections are required to participate and approve each issue.
\begin{itemize}
\item {} 
If passes, issue is merged into integration branch (developer marks the issue as done)

\item {} 
If fails, repeat process at step 3

\end{itemize}

\end{enumerate}


\section{Defect Reporting}
\label{\detokenize{test_plan/issue_management:defect-reporting}}

\subsection{Bug Life-cycle}
\label{\detokenize{test_plan/issue_management:bug-life-cycle}}
\noindent\sphinxincludegraphics[width=600\sphinxpxdimen]{{defect_reporting}.png}
\begin{enumerate}
\sphinxsetlistlabels{\arabic}{enumi}{enumii}{}{.}%
\item {} 
Bug or defect is reported and the founder will estimate the priority and severity

\item {} 
The bug will be verified by the team

\item {} 
The priority and severity of the bug will be verified

\item {} 
The bug will be assigned to a developer based on who’s code the bug was found in.

\item {} 
The developer fixes the bug.

\item {} 
The fix will be merged back to the integration branch of the project.

\item {} 
Regression tests will be run to ensure the bug was fixed.

\item {} 
Defect is closed.

\end{enumerate}


\subsection{Severity Levels}
\label{\detokenize{test_plan/issue_management:severity-levels}}

\begin{savenotes}\sphinxattablestart
\centering
\begin{tabular}[t]{|\X{10}{50}|\X{40}{50}|}
\hline
\sphinxstyletheadfamily 
Severity
&\sphinxstyletheadfamily 
Risks
\\
\hline
1 (Critical)
&\begin{itemize}
\item {} 
Defect causes the application to crash or hang

\item {} 
Corrupts application or system data

\item {} 
Consumes system resources to the point that other system processes are adversely affected

\end{itemize}
\\
\hline
2 (High)
&\begin{itemize}
\item {} 
Missing major application functionality without a workaround

\end{itemize}
\\
\hline
3 (Medium)
&\begin{itemize}
\item {} 
Missing minor application functionality without a workaround

\item {} 
Missing major application functionality with a workaround

\item {} 
Defect causes other features to be unavailable for review or testing

\end{itemize}
\\
\hline
4 (Low)
&\begin{itemize}
\item {} 
Minor feature not working as per requirements but functionality is testable using workaround

\end{itemize}
\\
\hline
\end{tabular}
\par
\sphinxattableend\end{savenotes}


\section{Metrics}
\label{\detokenize{test_plan/issue_management:metrics}}
Tracking progress and success of the tests for each test cycle. Delivered to Project Manager and Development Team by QA Point of Contact.
\begin{itemize}
\item {} 
Weekly Status Report: Includes weekly pass/fail/complete percentages. Identify and troubleshoot any defects in the Critical category which have persisted for over a week.

\item {} 
Sprint End Report: Compile trajectory graphs for defect lists broken out by status, severity, and age.

\end{itemize}


\section{Start and End Criteria}
\label{\detokenize{test_plan/issue_management:start-and-end-criteria}}\begin{itemize}
\item {} 
Start criteria detailed in the {\hyperref[\detokenize{test_plan/strategy:strategy}]{\sphinxcrossref{\DUrole{std,std-ref}{Strategy}}}} section.

\item {} 
Start criteria refer to the desirable and necessary conditions which need to be in place before test execution can be started.

\item {} 
Start and end criteria are flexible since, especially during Sprint 2, it is understood that environments, accounts, data, and documentation may still be in an immature state. Start criteria will be evaluated by Product Manager for a go no-go determination at the start of a sprint.

\item {} 
End criteria
\begin{itemize}
\item {} 
Test Script execution     \sphinxstylestrong{Owner: Dev Team}

\item {} 
95\% pass rate on Test Scripts             \sphinxstylestrong{Owner: Dev Team}

\item {} 
Zero severity 1 or 2 level defects        \sphinxstylestrong{Owner: Dev Team}

\item {} 
95\% severity 3 level defects closed \sphinxstylestrong{Owner: Dev Team}

\item {} 
Remaining defects converted to Change Requests or Deferred   \sphinxstylestrong{Owner: Dev Team}

\item {} 
100\% Coverage of requirements captured by expected and actual test script execution.                      \sphinxstylestrong{Owner: Dev Team}

\item {} 
100\% Test strategy metrics collected      \sphinxstylestrong{Owner: Dev Team}

\item {} 
100\% of defects logged in ZenHub’s  Issue Tracker System  \sphinxstylestrong{Owner: Dev Team}

\item {} 
Final Test report reviewed, verified, and signed off on by Product Manager and Dev Team

\item {} 
Test environment check pointed, tagged, and backed up     \sphinxstylestrong{Owner: Dev Team}

\end{itemize}

\end{itemize}


\chapter{Test Management}
\label{\detokenize{test_plan/test_management:test-management}}\label{\detokenize{test_plan/test_management::doc}}
Test Management is accomplished using a variety of tools. All testing artifacts, documents, issues, test cases, and results are stored, verified, and updated using the ZenHub Issue Tracking System.
\begin{itemize}
\item {} 
Developer technical communications including technical presentations, meeting minutes, and communications with the sponsor will be placed into Google Drive.

\item {} 
During test design, tests will be placed under revision control to ensure logging of change history.

\item {} 
Development Team members have access to individual test results and issue documentation.

\end{itemize}


\section{Test Design}
\label{\detokenize{test_plan/test_management:test-design}}\begin{itemize}
\item {} 
Team member reviews requirement under User Story and prepares tests which verifies requirement is met.

\item {} 
Test cases are mapped to User Stories and Requirements as part of requirement tracking.

\item {} 
Test cases are reviewed by Development Team to ensure the test faithfully validates existing requirement(s).

\item {} 
Development Team will use prototypes, user stories, use cases, and functional specifications to write step by step test cases.

\item {} 
QA Team will maintain test and issue tracking information to be shared periodically with Project Manager. Change requests or requirement clarifications can cause test cases to be modified, added, or removed as necessary.

\item {} 
Change requests must be reviewed and accepted by Development Team.

\end{itemize}


\section{Executing the Test Plan}
\label{\detokenize{test_plan/test_management:executing-the-test-plan}}\begin{itemize}
\item {} 
QA Team performs testing tasks as per test plan.

\item {} 
Defects are logged using the ZenHub Issue Tracking System. Developer to report the defect is responsible for initial assignment of severity but final determination made by the entire Development Team.

\item {} 
Product issues related to defects that prevent execution on test plan are reported, logged, and escalated as necessary to the Development Team. e.g. defects causing product features to be unavailable for testing.

\item {} 
Any defects marked as fixed in a previous test cycle are verified as fixed using test scripts and regression tests.

\end{itemize}


\section{Risks and Risk Response}
\label{\detokenize{test_plan/test_management:risks-and-risk-response}}

\begin{savenotes}\sphinxattablestart
\centering
\begin{tabular}[t]{|\X{20}{90}|\X{10}{90}|\X{30}{90}|\X{30}{90}|}
\hline
\sphinxstyletheadfamily 
Risk
&\sphinxstyletheadfamily 
Likelihood
&\sphinxstyletheadfamily 
Effect
&\sphinxstyletheadfamily 
Response
\\
\hline
Resource availability
&
Low
&
Unable to receive information we need to render a specific location. Users will no longer be able to use the application that requires information that we don’t currently have for the next 24 hours.
&
Ensure proper storage of repeatedly used testing resources, and understanding capabilities of API token reuse to minimize new API calls.
\\
\hline
Unforeseen Delays
&
Low
&
Impossible to tell the impact due to not knowing the issue. It could potentially create a time slip and we will not be able to deliver the product that we have agreed to produce for L3Harris, which may impact future projects with the company.
&
Work will be scheduled the highest priority in the next Sprint. During Christmas break there will be some time to make up for time slips that may occur to get the project back on schedule for development for Certification 3.
\\
\hline
\end{tabular}
\par
\sphinxattableend\end{savenotes}


\chapter{Test Environment and Product Requirements}
\label{\detokenize{test_plan/test_environment_and_product_requirements:test-environment-and-product-requirements}}\label{\detokenize{test_plan/test_environment_and_product_requirements::doc}}

\section{External Interfaces and Requirements}
\label{\detokenize{test_plan/test_environment_and_product_requirements:external-interfaces-and-requirements}}\begin{itemize}
\item {} 
Virtual Reality Headset (Desktop Based)

\item {} 
1 Gbps or better network connection

\end{itemize}


\section{Hardware}
\label{\detokenize{test_plan/test_environment_and_product_requirements:hardware}}\begin{itemize}
\item {} 
Quad Core Processor

\item {} 
8GB of RAM

\item {} 
GTX 970/RX 480

\item {} 
1 Gbps NIC

\item {} 
MySQL Database (either hosted on the network or on the local machine)

\end{itemize}


\section{Software}
\label{\detokenize{test_plan/test_environment_and_product_requirements:software}}\begin{itemize}
\item {} 
OS: Windows 10

\item {} 
MySQL Database

\end{itemize}


\chapter{Project Timeline}
\label{\detokenize{test_plan/project_timeline:project-timeline}}\label{\detokenize{test_plan/project_timeline::doc}}
\noindent\sphinxincludegraphics[width=630\sphinxpxdimen]{{project_plan}.png}


\chapter{Roles and Responsibilities}
\label{\detokenize{test_plan/roles_and_responsibilities:roles-and-responsibilities}}\label{\detokenize{test_plan/roles_and_responsibilities::doc}}

\section{Project Management}
\label{\detokenize{test_plan/roles_and_responsibilities:project-management}}
\sphinxstyleemphasis{Point of Contact: Jocelyne Freemyer}
\begin{itemize}
\item {} 
Liason for stakeholders

\item {} 
Assist in Sprint Planning

\item {} 
Facilitate team building activities

\item {} 
Review, verify, and confirm:
\begin{itemize}
\item {} 
User Stories

\item {} 
Test Plan

\item {} 
Test Strategy

\item {} 
Test Estimates

\end{itemize}

\end{itemize}


\section{QA Team}
\label{\detokenize{test_plan/roles_and_responsibilities:qa-team}}
\sphinxstyleemphasis{Point of Contact: Chezka Gaddi}
\begin{itemize}
\item {} 
Initial draft of test plan

\item {} 
Process for identifying, recording, and communicating defect reporting

\item {} 
Initial draft of issue reporting document (for review by Product Team and Dev Team)

\item {} 
Acknowledge and communicate test progress and completion for each test cycle

\item {} 
Give go-ahead for next test cycle at the completion of each cycle

\item {} 
Perform exploratory testing and report, develop, and communicate observed inconsistencies, gaps, or ambiguous requirements

\item {} 
Execute tests (test scripts, automated tests, and regression tests as needed)

\item {} 
Identify, record, and report defects

\end{itemize}


\section{Development Team}
\label{\detokenize{test_plan/roles_and_responsibilities:development-team}}
\sphinxstyleemphasis{Point of Contact: Isaac Egermier}
\begin{itemize}
\item {} 
Review test plan, burndown charts, test scripts, exploratory findings, automated tests, etc.  Sign-off or facilitate modifications as appropriate.

\item {} 
Deliver agreed upon product components as per scheduled dates

\item {} 
Communicate barriers to the schedule or product features in a timely manner

\item {} 
Implement fixes to defects discovered

\end{itemize}


\chapter{Team Members and Contact Info}
\label{\detokenize{test_plan/team_members_and_contact_info:team-members-and-contact-info}}\label{\detokenize{test_plan/team_members_and_contact_info::doc}}

\begin{savenotes}\sphinxattablestart
\centering
\begin{tabular}[t]{|\X{23}{63}|\X{15}{63}|\X{25}{63}|}
\hline
\sphinxstyletheadfamily 
Position
&\sphinxstyletheadfamily 
Name
&\sphinxstyletheadfamily 
Contact Info
\\
\hline
Project Manager
&
Jocelyne Freemyer
&
\sphinxhref{mailto:Jocelyne.Freemyer@mines.sdsmt.edu}{Jocelyne.Freemyer@mines.sdsmt.edu}
\\
\hline
Development Point of Contact
&
Isaac Egermier
&
\sphinxhref{mailto:Isaac.Egermier@mines.sdsmt.edu}{Isaac.Egermier@mines.sdsmt.edu}
\\
\hline
Quality Assurance Point of Contact
&
Chezka Gaddi
&
\sphinxhref{mailto:Chezka.Gaddi@mines.sdsmt.edu}{Chezka.Gaddi@mines.sdsmt.edu}
\\
\hline
Repo Manager
&
Michael Theesen
&
\sphinxhref{mailto:Michael.Theesen@mines.sdsmt.edu}{Michael.Theesen@mines.sdsmt.edu}
\\
\hline
Architecture Point of Contact
&
Garfield Tong
&
\sphinxhref{mailto:Garfield.Tong@mines.sdsmt.edu}{Garfield.Tong@mines.sdsmt.edu}
\\
\hline
\end{tabular}
\par
\sphinxattableend\end{savenotes}


\chapter{Sign Off}
\label{\detokenize{test_plan/sign_off:sign-off}}\label{\detokenize{test_plan/sign_off::doc}}

\begin{savenotes}\sphinxattablestart
\centering
\begin{tabular}[t]{|\X{13}{115}|\X{20}{115}|\X{22}{115}|\X{20}{115}|\X{20}{115}|\X{20}{115}|}
\hline

Name:
&
Isaac Egermier
&
Jocelyne Freemyer
&
Chezka Gaddi
&
Michael Theesen
&
Garfield Tong
\\
\hline
&&&&&\\
Signature:
&&&&&\\
&&&&&\\
\hline
Date:
&&&&&\\
\hline
\end{tabular}
\par
\sphinxattableend\end{savenotes}

\appendix 
\addcontentsline{toc}{chapter}{APPENDICES}

\chapter{Repository Management}

The repository is owned by Michael Theesen and is stored in the GitHub server under the name VRtualize-Pangeo. All members of the team have Developer permissions.


\subsection{Branch Structure}
\label{\detokenize{test_plan/repository_management:branch-structure}}
\noindent\sphinxincludegraphics[width=600\sphinxpxdimen]{{branch_structure}.png}


\subsubsection{Feature Branches}
\label{\detokenize{test_plan/repository_management:feature-branches}}
Each issue with have a feature branch for the purpose of developement specifically for that issue named using the following standard:

\sphinxstyleemphasis{developer-name}/\sphinxstyleemphasis{issue\#-name-of-issue}

Unit tests will be developed, run, and passed in this branch before a code review can be scheduled.


\subsubsection{Integration Branch}
\label{\detokenize{test_plan/repository_management:integration-branch}}
\sphinxtitleref{Cost of Entry}: Passes unit tests and code review.

The relationships between units of code is tested within this branch.

Advancement to upper branches will not occur until the code base of Certification 2 is in the integration branch.


\subsubsection{Testing Branch}
\label{\detokenize{test_plan/repository_management:testing-branch}}
\sphinxtitleref{Cost of Entry}: Passes integration tests (Certification 2)

The application as a whole is tested within this branch.


\subsubsection{Master}
\label{\detokenize{test_plan/repository_management:master}}
\sphinxtitleref{Cost of Entry}: Passes all tests.

Product-ready code is found in this branch.



\renewcommand{\indexname}{Index}
\printindex
\chapter{Issue Tracking Manager}

ZenHub is an agile project management application within GitHub that allows organization of issues into task boards and reports.


\subsection{Installation}
\label{\detokenize{test_plan/issue_tracking_manager:installation}}
Download the browser extension for Firefox or Chrome and accept permissions and sign in with GitHub. ZenHub will be added right into the GitHub interface.


\subsection{Structure}
\label{\detokenize{test_plan/issue_tracking_manager:structure}}
The ZenHub Workspace is split up between 7 pipelines;
\begin{itemize}
\item {} \begin{description}
\item[{New Issues}] \leavevmode
Issues that have yet to be assigned or evaluated.

\end{description}

\item {} \begin{description}
\item[{Icebox}] \leavevmode
Features that are desired but not required.

\end{description}

\item {} \begin{description}
\item[{Backlog}] \leavevmode
Issues that were not completed from the past Sprint. The highest priority.

\end{description}

\item {} \begin{description}
\item[{In Progress}] \leavevmode
Issues that currently being worked on by the Development team.

\end{description}

\item {} \begin{description}
\item[{Review/QA}] \leavevmode
Issues that are ready to be code reviewed and QA tested.

\end{description}

\item {} \begin{description}
\item[{Done}] \leavevmode
Issues that have passed review and QA tests and can be queued for next release.

\end{description}

\item {} \begin{description}
\item[{Closed}] \leavevmode
Issues that will no longer be worked on (Done, won’t fix, done with development, already implemented, duplicate, etc.).

\end{description}

\end{itemize}


\subsection{Issue Types}
\label{\detokenize{test_plan/issue_tracking_manager:issue-types}}\begin{itemize}
\item {} 
User Story

\item {} 
Feature Request

\item {} 
UX Issue

\item {} 
Performance

\item {} 
Bug

\end{itemize}


\subsection{Issue Complexities}
\label{\detokenize{test_plan/issue_tracking_manager:issue-complexities}}

\begin{savenotes}\sphinxattablestart
\centering
\begin{tabulary}{\linewidth}[t]{|T|T|}
\hline
\sphinxstyletheadfamily 
Complexity
&\sphinxstyletheadfamily 
Difficulty
\\
\hline
1
&
Trivial (Typo, small fix)
\\
\hline
2
&
Busy Work (Easy Function)
\\
\hline
3
&
Rare (New Function)
\\
\hline
5
&
Medium Rare (Set of interacting functions)
\\
\hline
8
&
Medium Well (New feature)
\\
\hline
13
&
Well Done (Major fix, multiple features)
\\
\hline
21
&
Perplexing (Researching new techniques)
\\
\hline
40
&
Beastly (Unsure if possible)
\\
\hline
\end{tabulary}
\par
\sphinxattableend\end{savenotes}



\renewcommand{\indexname}{Index}
\printindex


\renewcommand{\indexname}{Index}
\printindex
\end{document}